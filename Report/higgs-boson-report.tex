\documentclass[]{article}
\usepackage[a4paper,top=3cm,bottom=3cm,left=3cm,right=3cm]{geometry}

\usepackage{amsmath}
\usepackage{mathtools}
\usepackage{bm}
\usepackage{amsfonts}
\DeclareMathOperator*{\argmin}{argmin}
\DeclareMathOperator*{\argmax}{argmax}
\usepackage{graphicx}
\usepackage[colorlinks=true, allcolors=blue]{hyperref}
\usepackage{amsthm}
\usepackage{enumitem}
\usepackage{physics}
\usepackage{tabularx}
\usepackage{float}

\newtheorem{theorem}{Theorem}[section]
\newtheorem{corollary}{Corollary}[theorem]
\newtheorem{lemma}[theorem]{Lemma}
\newtheorem{remark}[theorem]{Remark} 

%opening
\title{Higgs Boson Machine Learning Project}
\author{Ettore Fincato, Hannah Sansford, Harry Tata}

\begin{document}

\maketitle

\begin{abstract}

Write abstract here.

\end{abstract}

\section{Introduction}

\subsection{Background}

The Higgs boson can \textit{decay} through various different processes, producing other particles in the process. In physics, one calls a decay into specific particles a \textit{channel}. Until fairly recently, the Higgs boson had been seen only in boson pair decay channels. It is now of importance to seek evidence on the decay into \textit{fermion} pairs, specifically \textit{tau-leptons} or \textit{b-quarks}, and to measure their characteristics \cite{higgs-challenge}. The ATLAS experiment \cite{ATLAS-experiment} was the first to report evidence of the \textit{H} to tau-tau channel and the goal of this report is to improve on this analysis.


\subsection{Overview}

The Atlas experiment at CERN provided simulated data that was used to optimise the analysis of the Higgs boson. In the Large Hadron Collider (LHC), proton bunches are accelerated in both directions on a circular trajectory. This results in some of the protons colliding as the bunches cross the ATLAS detector (called an \textit{event}), which produces hundreds of millions of proton-proton collisions per second. The particles resulting from each event are detected by sensors and, from this raw data, certain real-valued features are estimated \cite{higgs-challenge}.

Most of the uninteresting events (called the \textit{background}) are discarded using a real-time multi-stage cascade classifier. However, many of the remaining events represent known processes that are also known as \textit{background}. Our aim is to find the region of the feature space in which there is a significant excess of events compared to what known background processes can explain (called \textit{signal}).

Once the region has been fixed, the significance of the excess is determined using a statistical test. If the probability that the excess has been produced by background processes falls below a pre-determined limit, the new particle is deemed to be discovered.




\section{Problem Formulation}

Let $\mathcal{D} = \{(\bm{x}_1, y_1,w_1),...,(\bm{x}_n,y_n,w_n)\}$ ve the training set, where $\bm{x}_i \in \mathbb{R}^d$ is a $d$-dimensional feature vector, $y_i \in \{\text{b,s}\}$ is the label, and $w_i \in \mathbb{R}^+$ is a non-negative weight. Let $\mathcal{S} = \{i : y_i = \text{s}\}$ and $\mathcal{B} = \{i : y_i = \text{b}\}$ be the index sets of signal and background events respectively, and let $n_\text{s} = |\mathcal{S}|$ and $n_\text{b} = |\mathcal{B}|$ be the number of simulated signal and background events. 

The simulated dataset also includes importance weights for each event. Since the objective function (\ref{AMS}) depends on the \textit{unnormalised sum} of weights, in order to make the setup invariant to the \textit{numbers} of simulated events $n_s$ and $n_b$, the sum across each set (test/training) and each class (signal/background) is set to be fixed, i.e.,
\begin{equation}
	\sum_{i \in \mathcal{S}} w_i = N_\text{s} \hspace{10pt} \text{and} \hspace{10pt} \sum_{i \in \mathcal{B}} w_i = N_\text{b}.
\end{equation}
These normalisation constants $N_\text{s}$ and $N_\text{b}$ are simply the \textit{expected total number} of signal and background events, respectively, during the time interval of the data taking. The individual weights are then proportional to the conditional densities,
$$ p_\text{s}(\bm{x}_i) = p(\bm{x}_i|y=\text{s}) \hspace{10pt} \text{and} \hspace{10pt} p_\text{b}(\bm{x}_i) = p(\bm{x}_i|y=\text{b}), $$
divided by the instrumental densities $q_\text{s}(\bm{x}_i)$ and $q_\text{b}(\bm{x}_i)$, i.e.,
\begin{equation}
	w_i \propto \begin{cases}
	p_\text{s}(\bm{x}_i)/ q_\text{s}(\bm{x}_i), & \text{if} \hspace{10pt} y_i= \text{s}. \\
	p_\text{b}(\bm{x}_i) / q_\text{b}(\bm{x}_i), & \text{if} \hspace{10pt} y_i= \text{b}.
	\end{cases}
\end{equation}


Now, let $g: \mathbb{R}^d \to \{\text{b,s}\}$ be a classifier. Let the \textit{selection region} $\mathcal{G} = \{\bm{x}: g(\bm{x}) = \text{s}\}$ be the set of points classified as signal, and let $\hat{\mathcal{G}}$ denote the \textit{index set} of points that $g$ classifies as signal, i.e.,
$$ \hat{\mathcal{G}} = \{i:f(\bm{x}_i) \in \mathcal{G}\} = \{i : g(\bm{x}_i) = \text{s}\}. $$
Then we have that
\begin{equation}
\label{s weights}
	s = \sum_{i \in \mathcal{S} \cap \hat{\mathcal{G}}} w_i
\end{equation}
is an unbiased estimator of the expected number of signal events selected by $g$, and similarly,
\begin{equation}
\label{b weights}
b = \sum_{i \in \mathcal{B} \cap \hat{\mathcal{G}}} w_i
\end{equation}
is an unbiased estimator of the expected number of background events selected by $g$. Alternatively, $s$ and $b$ are the \textit{unnormalised} true and false positive rates, respectively.

High-energy physicists suggest the use of the \textit{approximate median significance} (AMS) objective function defined by
\begin{equation}
\label{AMS}
	\text{AMS} = \sqrt{2 \left[ (s + b + b_\text{reg}) \ln \left( 1 + \frac{s}{b + b_\text{reg}} \right) - s \right]}
\end{equation}
to optimise the selection region for discovery significance, where $b_\text{reg}$ is a regularisation term suggested to be set to $b_\text{reg}=10$. Hence, our aim is simply to train a classifier $g$ based on the training data $\mathcal{D}$ with the goal of maximising the AMS on some unseen test data.

\section{Maximising the AMS}

Having decided a statistical model for classification, its parameters will be tuned by maximising the AMS. This will be done in two ways.

\begin{itemize}
\item \textbf{Direct maximisation}: One can consider the AMS as an error/objective measure in a cross validation procedure. So the model will be tuned by a cross validation (cv) procedure by looking at the coefficients which directly maximise the AMS.

\item \textbf{Two-stage maximisation}: Given a real-valued function $f$ (e.g. a linear $f(w,x)=w'x$), one can always get a classifier by setting a threshold $\theta \in A\subset \mathbb{R}$ and a transformation $g$ and considering
\begin{equation}
\label{classifier from f}
h(\theta):=\operatorname{sgn}(g(f)-\theta)\in \{-1,1\}
\end{equation}
One can therefore train and tune \textit{a suitable transformation $g$ of} $f$ "independently" of the AMS, and then find the threshold $\theta$ for which the classifier $h(\theta)$ maximises the AMS. This is based on \cite{kotlowski2014consistent}
\end{itemize}

\subsection{Two-stage maximisation}

While the direct maximisation with cv can be seen as a usual tuning procedure by cross validation, the two-stage maximisation should be justified. This section is dedicated to the explanation of the two stage procedure for a  classifier $h:\mathcal{X}\to \{-1,1\}$ defined as in \ref{classifier from f} , where $\mathcal{X}$ denoted the feature espace and \textbf{the suitable transformation $g$ is given by the logistic loss} of $f$. \\

By incorporating the weights to the probabilities, without loss of generality one may write  $s$ and $b$ from \ref{s weights} and \ref{b weights} as

$$s=s(h):=\mathbb{P}(h(X)=1,Y=1) \ \text{,} \ \ \ b=b(h):=\mathbb{P}(h(X)=-1,Y=1)$$

where $X$ denotes a random imput vector and $Y$ a random binary response. The paper justifies the two stage procedure explained above for logistic loss transformation of $f$, as follows. Given a classifier $h$ obtained from $f$ as in \ref{classifier from f}, one can define the "AMS regret" as
\begin{equation}
\label{regret}
R_{AMS}(h):= AMS^2(h^*)-AMS^2(h)
\end{equation}
where $$h^*:=\underset{h}{\text{argmax}}AMS^2(h).$$

Similarly, the "logistic regret" of $f$ is defined as
\begin{equation}
\label{logistic regret}
R_{log}(f):= L(f)-L(f^*)
\end{equation}
for the expected logistic loss transformation of $f$ $$L(f):=\mathbb{E}\Big[\log(1+Ye^{-f(X)})\Big]$$ and $$f^*:=\underset{f}{\text{argmax}}L(f)$$

The formal justification of the idea of \begin{itemize}
\item  Training a logistic loss transformation of $f$ and
\item Finding the threshold $\theta$ in \ref{classifier from f} by maximising the AMS on a second stage
\end{itemize}
comes from the following theorem.

\begin{theorem}
\label{thm: two stage maximisation}
Given a real-valued function $f$ and the related classifier $h(\theta):=\operatorname{sgn}(f-\theta)$,  let $\hat{\theta}:=\underset{\theta}{\operatorname{argmax}} AMS(h)$. Then,
\begin{equation}
\label{regret inequality}
R_{AMS}(h(\theta))\leq \frac{s(h^*)}{b(h^*)}\sqrt{\frac{1}{2}R_{log}(f)}
\end{equation}

\end{theorem}
\begin{proof}
 See \cite{kotlowski2014consistent}-Theorem 2.
\end{proof}

That is, the AMS regret of the classifier $h(\theta)$ is upper bounded by the logistic regret of the underlying $f$. It is possible to prove that it is sufficient to optimize $\theta$ on the empirical counterpart of AMS calculated on a separate validation sample. 


\begin{remark}
The use of a logistic loss transformation of $f$ is particularly suited for a classification task. For example, it allows us to train (on the first stage) models such as a logistic regression. 
\end{remark}


\bibliographystyle{plain}
\bibliography{refs}

\end{document}
